\documentclass[a4]{report}
\usepackage{graphicx}
\usepackage{listings} % TODO
\usepackage{svg}
\usepackage[table]{xcolor}
\usepackage{caption}
\usepackage{mdframed}
\usepackage{subcaption}
\usepackage[a4paper, top=1.5cm, bottom=1.5cm]{geometry}
\usepackage{adjustbox}
\usepackage[
backend=biber,
style=numeric,
sorting=ynt
]{biblatex}
\addbibresource{bibliography.bib}

\title{IMAP Client with TLS support}
\author{Iaroslav Vasilevskii}

\newcommand{\TODO}{
    \textbf{\textcolor{red}{TODO}}
}

\begin{document}

\maketitle
\tableofcontents

\chapter{Introduction}

{\Large\textbf{imapcl}} is an IMAP4rev1\cite{rfc3501} client able to fetch messages from a server. Besides plain-text communication, it also supports communication secured by TLS with the help of the OpenSSL library.

\section{IMAP}

\textbf{IMAP} Version 4rev1 is an application-level protocol that allows for access and manipulation of electronic mail messages on a server. In particular, it supports creating, deleting, and renaming mailboxes, checking for new messages, removing messages and manipulating flags\cite{rfc3501}.

\section{SSL/TLS}

\textbf{TLS} (Transport Layer security) allows client/server applications to communicate over the Internet in a way protected against eavesdropping, tampering, and message forgery\cite{rfc8446}.

TLS was introduced as a replacement for the SSL 2.0 protocol, which was discovered to be seriously vulnerable, leading to it being abandoned\cite{rfc6176}.

\subsection{Asymmetric cryptography}

\textbf{Asymmetric cryptography} refers to cryptographic methods and algorithms that do not require that the two counterparts share the same key pre-shared over a secure channel, but the key is instead shared during the key exchange process over an insecure channel(e.g. Diffie-Hellman Key Exchange). TLS/SSL can be configured for use both with the key exchange algorithm, or with a pre-shared key (TLS-PSK)\cite{8446}.

\subsection{Connection initiation}

TLS works over TCP, hence first a TCP handshake is performed, followed by a TLS handshake. During the TLS handshake, the counterparts negotiate the connection according to their supported protocol versions and cipher suites, exchange and verify each other's certificates against the \textit{authorities} whom the verifying side trusts.

\begin{enumerate}
  \item Client sends a \texttt{Client Hello} message to the server, containing its highest \textit{TLS Version} supported, a list of supported \textit{cipher suites}, a \textit{random number} for key generation, followed by standardized extensions and other fields.
  \item Server sends a \texttt{Server Hello} message to the client, including, among other information, the selected \textit{TLS Version}, and the server's \textit{random number}.
  \item The server sends its certificate to the client, and the client may send its certificate to the server upon request to check each other's identity against an authority.
\end{enumerate}

The server may also request additional information from the client instead of sending a \texttt{Server Hello} (a \texttt{Hello Retry Request}), if the server requires additional information\cite{rfc8446}.

\subsection{Certificates and CAs}

A \textbf{certificate} is an authorization token that is signed by a \textit{Certification Authority} (CA), typically using an algorithm like RSA, DSA, or, more recently, ECDSA.

The signature algorithm based on \textbf{RSA} functions as follows (Alice wants to send a signed message to Bob; $d$, $e$ are Alice's private and public keys respectively., $n=pq$ is the modulo; Bob has Alice's public key $e$):

\begin{enumerate}
  \item Alice computes $s=hash(m)^{d}\pmod{n}$ and sends it along with $m$.
  \item Bob computes $k=s^{e} \pmod{n}$ and compares if $k = hash(m)$. If it is so, then Bob can be sure that it was indeed Alice who sent the message, because $(h^{e})^{d}=(h^{d})^{e} \pmod{n}$.
\end{enumerate}

A \textbf{Certification Authority} (CA) is a party which is trusted by a group of systems. That is, certificates signed by a CA are viewed by these systems as having the weight of a proof of the counterpart's identity when communicating.

\chapter{Design}


\section{Response Parsing}

\textbf{IMAP} is a comparatively complex protocol, and its context-free grammar asks for a more advanced parsing method than mere regular expression matching. Recursive descent parsers, on the other hand, are widely used for analyzing context-free languages, including network protocols. The parser implementation thus turned out straightforward, albeit requiring some care and debugging.

\section{Messages}

Messages are represented in the design as classes that inherit from either \texttt{Command} or \texttt{Response} abstract classes. Every command has a tag, which is empty for non-tagged commands, and a type, which identifies its role in the protocol. Similarly, every \texttt{Response} has a type and some subclasses have tags. % TODO: Think if it's a correct design

\section{Message receipt}

The response receipt mechanism is designed as a class that notifies the \texttt{Session} object of any incoming message.

\section{Diagrams}

\begin{figure}
  \centering
  \includegraphics*[width=0.33\textwidth]{imapcl-state.png}
  \caption{State diagram}
  \label{state}
\end{figure}

\begin{figure}[t]
  \centering
  \rotatebox{90}{\includegraphics*[width=1.5\textwidth]{imapcl-class.png}}
  %%\rotatebox{90}{\includesvg[width=1.5\textwidth]{imapcl-class.svg}[h]} %% Inkscape doesn't read the file properly for some reason when called by latexmk
  \caption{Class diagram}
  \label{class}
\end{figure}


\chapter{Implementation}

The program is implemented in \textit{C++20}. In view of limited support for C++20 on the reference machine with \textit{GCC 11.5.0}, not all features of the standard could be used in the project. Nevertheless, the limitations proved to be insignificant to the project as a whole and caused minimal problems.

\section{Dependencies}

\begin{itemize}
\item \texttt{make}
\item \texttt{gcc}
\item \texttt{OpenSSL}
\item \texttt{googletest} \textit{(when building unit tests)}
\end{itemize}


\section{Project file structure}

\begin{itemize}
\item \texttt{Makefile} - Makefile managing the build process
\item \texttt{receiver.hpp, receiver.cpp} - Class implementing asynchronous receipt of messages
\item \texttt{tls\_receiver.hpp, tls\_receiver.cpp} - A subclass of \texttt{Receiver} supporting TLS
\item \texttt{command.hpp, command.cpp} - Command class representing commands sent by the user
\item \texttt{fnv.hpp, fnv.cpp} - Fowler-Noll-Vo\cite{eastlake-fnv-29} hashing function class for creating unique filenames
\item \texttt{logger.hpp, logger.cpp} - Class implementing logging capabilities with adjustable logging level
\item \texttt{main.hpp, main.cpp} - Main function and argument parsing
\item \texttt{message.hpp}
\item \texttt{parser\_logger.hpp, parser\_logger.cpp} - Logger subclass for the recursive descent parser
\item \texttt{response.hpp, response.cpp} - Response class representing messages sent by the server
\item \texttt{response\_parser.hpp, response\_parser.cpp} - Class for parsing server response data
\item \texttt{server.hpp, server.cpp} - Class initializing the socket and managing message sending
\item \texttt{tls\_server.hpp, tls\_server.cpp} - Server subclass adding SSL/TLS initialization and management
\item \texttt{session.hpp, session.cpp} - Class managing user actions on a high level
\item \texttt{client.hpp, client.cpp} - A class implementing a REPL to interact with the user in interactive mode.
\item \texttt{imf\_message.hpp, imf\_message.cpp} - A class implementing parsing of some components of a message in IMF format.
\item \texttt{test/} - Directory with test files, further documented in \textit{Testing}.
\end{itemize}

\section{Elements}

\subsection{Response parser}

A handwritten backtracking recursive-descent parser is used to parse server responses according to the formal syntax specification in \cite{rfc3501}. The parser structure closely follows the formal specification of the language.

Even though backtracking is known to bear risk of bringing exponential time complexity to the parser, the implemented grammar is comparatively simple, and the messages short, so the amount of backtracking that can happen is limited.

\subsection{Session}

The \texttt{Session} class represents a connection to the IMAP server. It features a \textit{finite-state machine} for keeping track of the communication state and \textit{methods} for individual actions like \texttt{fetch}, \texttt{select}, \texttt{search} and \texttt{login}.

\subsection{Command}

The \texttt{Command} class contains \texttt{type} and \texttt{tag} protected members, with the tag being empty for non-tagged responses. They also have a \texttt{make\_tcp()} method, which marshals the command information into the form of a protocol message. Specialized commands contain additional fields and methods for working with relevant data.

\subsection{Response}

The response class has a \texttt{type}, as well as a \texttt{tag} (again, empty for non-tagged responses).

\subsection{Client}

The client class contains a constant structure with regular expressions for command matching, and a read-eval-print loop function that processes user commands.

\subsection{IMFMessage}

IMFMessage is a class for parsing of selected parts of a message in IMF format\cite{rfc5322}. It is only used to generate reasonable file names for downloaded messages. It can parse the \texttt{Date:} header in the message.

\subsection{Receiver}

Receiver is a class whose \texttt{receive} method runs in a separate thread and notifies the Session when there is a new incoming message. TLSReceiver uses TLS/SSL communication instead of plain sockets. The virtual \texttt{receive_inner} method implements the subclass-specific logic, containing the recv() or SSL_read() call. Receiver runs in an infinite loop until the \texttt{stopped} value is \texttt{true}. The socket timeout is set up so that it is possible to stop the receiver thread at any time; if recv() times out, the loop checks the \texttt{stopped} value and continues, not notifying the session. If some data was received, the session is immediately notified.

\subsection{Logger}

The \texttt{Logger} class has 3 methods for logging debug information, errors and general information accordingly.

The \texttt{ParserLogger} subclass also has the \texttt{parser\_debug\_log} method for visualizing current depth of call stack, method names, and current character index along with the character itself.

\subsection{Server}

The \texttt{Server} class has a buffer for incoming responses, and constructor which inicializes the connection, as well as a \texttt{receive()} method, returning the received response. The constructor throws an exception if the attempt to establish the connection is unsuccessful.

\section{Limitations}

\begin{itemize}
  \item When the program finishes and the destructor of \texttt{Session} is called, it can take up to 1 second before the receiver thread times out, leaves uninterruptible sleep, and is joined.
\end{itemize}

\chapter{Usage}

\section{CLI Arguments}

\begin{mdframed}
\begin{verbatim}
$ imapcl SERVER [-p port] [-T [-c certfile] [-C certaddr]] [-n] [-h]
-a auth_file [-b MAILBOX] -o out_dir
\end{verbatim}
\end{mdframed}

\begin{table}[h]
  \renewcommand*{\arraystretch}{1.0}
  \centering
  \begin{tabular}{|l|l|}
    \hline
    \rowcolor{lightgray}
    Option & Description\\
    \hline
    \texttt{-p PORT}      & Port to connect to. Default is 993 for IMAPS and 143 for plain IMAP.\\
    \texttt{-T}          & Enables encryption (use IMAPS)\\
    \texttt{-c FILE}     & Specify certificate file\\
    \texttt{-C DIR}      & Specify certificate directory (default /etc/ssl/certs)\\
    \texttt{-n}          & Only see new messages\\
    \texttt{-h}          & Only fetch headers\\
    \texttt{-a AUTH\_FILE}& File with login credentials to use\\
    \texttt{-b MAILBOX}  & Mailbox to use on the server (default INBOX)\\
    \texttt{-o OUT\_DIR}  & Directory to store downloaded messages\\
    \texttt{-i}          & Start the program in interactive mode\\
    \hline
  \end{tabular}
  \caption{Command line options description}
\end{table}

\section{Examples}

\begin{figure}[h!]
\begin{subfigure}{\textwidth}
\begin{mdframed}[backgroundcolor=lightgray]
\begin{verbatim}
# ./imapcl 127.0.0.1 -p 143 -a credentials -b Important -h -o outdir/
Fetched 5 messages (headers).
\end{verbatim}
\end{mdframed}
\caption{Fetching message headers}
\end{subfigure}
\vspace{0.5cm}

\begin{subfigure}{\textwidth}
\begin{mdframed}[backgroundcolor=lightgray]
\begin{verbatim}
# ./imapcl 127.0.0.1 -p 143 -a credentials -b Important -o outdir/
Fetched 5 messages.
\end{verbatim}
\end{mdframed}
\caption{Fetching entire messages}
\end{subfigure}
\vspace{0.5cm}

\begin{subfigure}{\textwidth}
\begin{mdframed}[backgroundcolor=lightgray]
\begin{verbatim}
# ./imapcl 127.0.0.1 -p 143 -a credentials -b Important -n -o outdir/
Fetched 1 unseen messages.
\end{verbatim}
\end{mdframed}
\caption{Fetching only unseen messages}
\end{subfigure}
\caption{Usage examples}
\end{figure}

\chapter{Testing}

\textbf{Example of running the tests} can be found in the appendix.

\section{Unit tests}

\begin{itemize}
\item \texttt{test/make\_tcp.cpp} - \texttt{make\_tcp()} tests for \texttt{Command} and its subclasses
\item \texttt{test/parse\_response.cpp} - Tests for the response parser
\item \texttt{test/imf\_message.cpp} - Tests for the \texttt{IMFMessage} class
\item \texttt{test/test\_main.cpp} - Initialization of the Google Test framework
\end{itemize}

\section{Testing results}

\begin{figure}
\begin{verbatim}
# make test
g++ -std=c++20 -g -Wall -Wenum-compare -Wenum-conversion -Wpedantic -ggdb -O0 -g -c
command.cpp -o command.o
g++ -std=c++20 -g -Wall -Wenum-compare -Wenum-conversion -Wpedantic -ggdb -O0 -g -c
test/parse_response.cpp -o test/parse_response.o
g++ -std=c++20 -g -Wall -Wenum-compare -Wenum-conversion -Wpedantic -ggdb -O0 -g -c
response_parser.cpp -o response_parser.o
g++ -std=c++20 -g -Wall -Wenum-compare -Wenum-conversion -Wpedantic -ggdb -O0 -g -c
parser_logger.cpp -o parser_logger.o
g++ -std=c++20 -g -Wall -Wenum-compare -Wenum-conversion -Wpedantic -ggdb -O0 -g -lgtest
test/test_main.o test/make_tcp.o command.o test/parse_response.o response_parser.o response.o
logger.o parser_logger.o -o imapcl-tests
# ./imapcl-tests
[==========] Running 11 tests from 2 test suites.
[----------] Global test environment set-up.
[----------] 4 tests from CommandMakeTcpTests
[ RUN      ] CommandMakeTcpTests.Login
[       OK ] CommandMakeTcpTests.Login (0 ms)
[ RUN      ] CommandMakeTcpTests.Select
[       OK ] CommandMakeTcpTests.Select (0 ms)
[ RUN      ] CommandMakeTcpTests.Search
[       OK ] CommandMakeTcpTests.Search (0 ms)
[ RUN      ] CommandMakeTcpTests.Fetch
[       OK ] CommandMakeTcpTests.Fetch (0 ms)
[----------] 4 tests from CommandMakeTcpTests (0 ms total)

[----------] 7 tests from ResponseParserParseTests
[ RUN      ] ResponseParserParseTests.LoginNoError
[       OK ] ResponseParserParseTests.LoginNoError (0 ms)
[ RUN      ] ResponseParserParseTests.GreetingParseCorrect
[       OK ] ResponseParserParseTests.GreetingParseCorrect (0 ms)
[ RUN      ] ResponseParserParseTests.LoginParseCorrect
[       OK ] ResponseParserParseTests.LoginParseCorrect (0 ms)
[ RUN      ] ResponseParserParseTests.LoginParseCorrect2
[       OK ] ResponseParserParseTests.LoginParseCorrect2 (0 ms)
[ RUN      ] ResponseParserParseTests.SelectResponseParseCorrect
[       OK ] ResponseParserParseTests.SelectResponseParseCorrect (0 ms)
[ RUN      ] ResponseParserParseTests.SelectResponseParseCorrect2
[       OK ] ResponseParserParseTests.SelectResponseParseCorrect2 (0 ms)
[ RUN      ] ResponseParserParseTests.SearchResponseParseCorrect
[       OK ] ResponseParserParseTests.SearchResponseParseCorrect (0 ms)
[----------] 7 tests from ResponseParserParseTests (0 ms total)

[----------] Global test environment tear-down
[==========] 11 tests from 2 test suites ran. (0 ms total)
[  PASSED  ] 11 tests.
#
\end{verbatim}
\caption{Running the tests}
\end{figure}


\chapter{Further development}

\begin{itemize}
  \item Currently, the user has to store the credentials in a plain-text file. It would be reasonable to support encrypted authorisation files (e.g. using gpg).
\end{itemize}

\printbibliography

\end{document}
